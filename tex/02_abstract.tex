%!TEX root = ../thesis.tex

\chapter*{Abstract}
\label{chap:abstract}
\addcontentsline{toc}{chapter}{\nameref{chap:abstract}}
The rapid advancement of robotics and autonomous systems has the potential to revolutionize various industries, including healthcare, transportation, and manufacturing. Central to these innovations is the ability to accurately model and understand complex dynamic environments, which enables seamless and safe interactions with the real world. This thesis addresses the critical challenge of developing robust data-driven simulations for intelligent systems, focusing on two key areas: the representation and estimation of scene dynamics, and neural scene reconstruction. By leveraging data-driven approaches, we aim to overcome the limitations of traditional simulation methods, providing more realistic and varied scenarios for training and testing autonomous systems.

In this thesis, we make significant contributions to the field of data-driven neural simulation. \cref{chap:cvpr21} tackles the challenge of point cloud registration, especially for pairs with low overlap, by introducing the concept of the overlap region at the keypoint sampling step. This innovation significantly enhances the reliability of registration models through the novel overlap attention block. \cref{chap:iccv23} presents a powerful neural scene reconstruction method for data-driven LiDAR simulation, combining the robust representation capabilities of neural fields with a detailed physical sensor model. This approach enhances the realism of simulated LiDAR scans, facilitating their adoption for closed-loop testing of autonomous navigation systems. Recognizing the dynamic nature of real-world environments, \cref{chap:eccv22} introduces PCAccumlation, an efficient representation of scene dynamics that decomposes scenes into a static background and rigidly moving agents, using compact \textbf{SE(3)} transformations. Finally, \cref{chap:cvpr24} integrates this motion representation with neural scene reconstruction tailored for LiDAR simulation, creating a versatile neural simulator that enables various scene editing capabilities. These contributions provide essential building blocks for developing robust and reliable autonomous systems, promising to transform multiple industries by improving efficiency, safety, and overall quality of life.


\chapter*{Kurzfassung}
\label{chap:kurzfassung}
\addcontentsline{toc}{chapter}{\nameref{chap:kurzfassung}}
\foreignlanguage{german}{
german text
}