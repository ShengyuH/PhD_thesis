\chapter{Introduction} 
\section{Motivation}
% why do we fancy a world with many autonomous systems
Imagine waking up to a world where a service robot has already prepared your cappuccino and croissant, and an autonomous vehicle is ready to smoothly take you to your office. The advancements in robotics and autonomous systems have the potential to drastically revolutionize various aspects of our lives and numerous industries, including healthcare, transportation, manufacturing, and logistics. At the heart of these innovations is the capability to accurately model and understand complex physical interactions within dynamic environments. These sophisticated models are essential for developing intelligent systems that can interact seamlessly and safely with the real world.

% simulation plays a very important role in creating autonomous system
Simulation plays a critical role in the development of these intelligent systems. By providing controlled and authentic simulated environments, simulation allows these systems to be trained and rigorously tested. This approach is particularly important for exploring scenarios that are either too rare or too dangerous to reproduce in real life. For example, in autonomous driving, simulating scenarios such as sudden pedestrian crossings, unpredictable behavior of other vehicles, or severe weather conditions like snowstorms and heavy rain is crucial. Similarly, in healthcare, robots performing delicate surgeries can be trained in simulations to handle unexpected complications without risking patient safety. By simulating such scenarios, developers can ensure that their systems can handle a wide range of conditions and challenges, thus enhancing the overall robustness and reliability of autonomous systems. In this way, simulation not only accelerates the development process but also contributes to the safety and effectiveness of the deployed systems.

% traditional simultor is time-consuming and costly, can cause domain-gap 
Traditional simulation approaches, such as those proposed in CARLA~\cite{dosovitskiy2017carla} and Isaac Sim~\cite{makoviychuk2021isaac}, have significantly contributed to advancing autonomous systems. However, these methods come with several limitations. They are often costly and time-consuming to create, requiring extensive manual setup and calibration. This not only limits their scalability but also introduces a domain gap—the discrepancies between simulated environments and the real world. Such gaps can diminish the effectiveness of training, as models developed in simulated settings may not perform well when deployed in real-world scenarios. Consequently, while traditional simulations provide valuable insights and testing grounds, their limitations necessitate more efficient and scalable solutions to bridge the gap between simulation and reality.

% the alternative would be data-driven simulation
An alternative to traditional simulation methods is data-driven simulation, which leverages sensory data captured from real environments. This approach involves reconstructing the environment from the collected data and then simulating new sensory data by altering sensor configurations or modifying the environment itself. By utilizing real data, data-driven simulation significantly enhances the diversity of training and testing datasets, providing more realistic and varied scenarios for system training. This not only bridges the gap between simulation and reality but also ensures that the models are exposed to a wider range of conditions. Furthermore, data-driven simulations are generally more cost-effective and scalable compared to traditional methods, making them a promising solution for developing robust and reliable autonomous systems.

% what's the core to such data-driven simualtion.
The core of data-driven simulation consists of two main components: understanding and reconstructing scene dynamics, and simulating new sensory data based on this reconstruction. Dynamic scenes are typically captured by moving robots and can often be decomposed into a few moving objects and a static background. Understanding the motion of the robot or vehicle itself (ego-motion) and the motion of the dynamic parts from sensory data is the first step toward dynamic reconstruction. Accurately estimating these motions ensures that the simulation reflects the true dynamics of the real world.

Once the motion reconstruction is achieved, the next step involves simulating new sensory data by altering the sensor configurations or modifying the scene dynamics. This capability allows for the creation of diverse and realistic training scenarios by changing the sensor rig or editing the scene components. By simulating different viewpoints and conditions, this approach enhances the robustness and adaptability of the trained models, ultimately leading to more reliable and versatile autonomous systems.

\section{Reseach Questions}
In this thesis, we aim to advance the field of data-driven simulation by addressing two key aspects: multi-body dynamics reconstruction and compositional simulation. The following research questions are the focal points of our investigation:

\noindent
\textbf{\textit{Research Question 1: How can we reliably and accurately align two point cloud fragments that are sparsely captured?}}

Point cloud registration has been a persistent challenge. Although recent works~\cite{gojcic2018learned,Choy2019FCGF} have made significant strides on traditional benchmarks like 3DMatch~\cite{zeng20163dmatch}, their performance in challenging low-overlap scenarios remains suboptimal. This is particularly critical since real-world scenes are often captured sparsely, making low-overlap a common issue. Our goal is to develop a model that can robustly align point cloud fragments under low-overlap conditions.

\noindent
\textbf{\textit{Research Question 2: What is an effective motion representation for densely-captured dynamic street scenes, and how can we accurately estimate it?}}

Street scenes are frequently captured by RGB cameras and LiDAR sensors at high frequencies, comprising a static background, parked vehicles, and moving agents such as vehicles, bicycles, and pedestrians. Understanding these scene dynamics is essential for reconstruction and simulation. Previous works~\cite{li2020neural,gojcic2021weakly} either use unconstrained scene flow representations that may produce physically implausible motions or rely on only two frames, which do not account for the temporal smoothness of motions. Our objective is to design a method that handles multiple frames and accurately predicts physically viable motions.

\noindent
\textbf{\textit{Research Question 3: What is an appropriate neural scene representation and forward model for LiDAR simulation?}}

After addressing the motion estimation for both sparsely captured static scenes and densely captured dynamic scenes, the next step is to design a scene representation and forward model for simulation. Prior work~\cite{manivasagam2020lidarsim} employs explicit surface reconstruction and ray-surfel casting for LiDAR scans simulation. However, explicit reconstruction often results in the loss of surface details, and such decoupled two-stage methods cannot directly optimize the scene reconstruction for simulation tasks. Our goal is to create an alternative neural scene representation and a forward model tailored to LiDAR sensors to faithfully reconstruct and simulate LiDAR scans.

\noindent
\textbf{\textit{Research Question 4: How can we simulate LiDAR data in dynamic scenes?}}

Building on the neural scene representation and forward model developed for data-driven LiDAR simulation, the next step is to extend these methods to handle dynamic scenes. This involves integrating dynamic scene modeling, as discussed in Research Question 2, with the neural scene representation and forward model from Research Question 3. Our aim is to develop a comprehensive framework that can accurately simulate LiDAR data in dynamic environments.


\section{Outline \& Contributions} 
This thesis addresses the research questions outlined earlier and contributes to the instigation of scene dynamics reconstruction and simulation. Specific contributions are detailed as follows.


\subsection{Registration of 3D Point Clouds with Low Overlap}

\paragraph{Contributions}
\begin{itemize}
\item 
\end{itemize}

\subsection{Reconstruction of Multi-body Dynamics in Driving Scenes}

In Chapter ...

\paragraph{Contributions}
\begin{itemize}
\item A
\end{itemize}

\subsection{Neural LiDAR Fields for Novel View Synthesis}

In Chapter ...

\paragraph{Contributions}
\begin{itemize}
\item A
\end{itemize}


\subsection{Dynamic LiDAR Simulation using Compositional Neural Fields}

In Chapter ...

\paragraph{Contributions}
\begin{itemize}
\item A
\end{itemize}

Besides the above four papers included in this thesis, I also contribute the following works that are highly relevant to the thesis and also advance these two directions:
- Living scene
- Nothing Stands Still benchmark
- Relighting work
- Implicity





\section{Relevance to Science and Economy}

xxx

\paragraph{Open Source}
xxx

\paragraph{Scientific Recognition}
xxx