\chapter{Background}
\label{chap:background}
In this chapter, we provide a background overview of the research fields related to this thesis. Here for general interest, we specifically focus on two aspects: pointc cloud processing~\cref{sec:bg_point_cloud} and neural filed in visual computing~\cref{sec:bg_neural_field}. Later in each chapter, we dive into detailed related works for each specific task.


\section{Point cloud processing}
\label{sec:bg_point_cloud}
3D data can be represented using various forms, including multi-view images~\cite{su2015multi}, dense voxels~\cite{choy20163d}, point clouds~\cite{qi2017pointnet,qi2017pointnet++}, polygon mesh~\cite{hanocka2019meshcnn}, and implicit representations~\cite{niemeyer2019occupancy,park2019deepsdf,chen2019learning}. In this section, we introduce various learning-based methods that handle the unstructured point cloud data. 

\paragraph{Point-wise MLP}
DeepSets~\cite{zaheer2017deep} and PointNet~\cite{qi2017pointnet} poineered deep learning for point cloud processing by intriducing the permutation-equivariance operator. Taking PointNet for example, it first applies point-wise MLPs to extract high dimensional local information, then it uses the pooling operation along point dimension to reduce $N$ features to one global feature representing the entire point cloud, here $N$ demotes the number of points in this point cloud data. In this way, the model is invariant to arbitrary permutation of the un-ordered point cloud. PointNet is effective in modelling simple object but is inherently limited to handling complex geometry. This is because it only applies one global pooling operation to aggreated global information. PointNet++~\cite{qi2017pointnet++} addresses this limitation by repeatedly subsampling point cloud and appling pooling operation within each small neighborhood to aggregate for fine-grained information. 

\paragraph{Point convolution}



\paragraph{Sparse convolution}

\paragraph{Birds-eye-view projection}

\paragraph{Other representations and processing techniques}


\section{Neural field in visual computing}
\label{sec:bg_neural_field}




% \section{Point cloud Registration}
% \label{sec:bg_registration}


% \paragraph{Feature-based methods}
% \paragraph{Regression-based methods}
% \paragraph{Correspondence-free methods}

% \section{Scene flow estimation}
% \label{sec:bg_scene_flow}

% \paragraph{Soft rigidity}
% \paragraph{Hard rigidity}
% \paragraph{Optimisation-based methods}



\paragraph{What is neural fields}
\paragraph{Forward model in neural rendering}