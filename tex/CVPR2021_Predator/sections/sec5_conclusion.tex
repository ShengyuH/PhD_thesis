\section{Conclusion}
We have introduced \acro, a deep model designed for pairwise registration of low-overlap point clouds. The core of the model is an overlap attention module that enables early information exchange between the point clouds' latent encodings, in order to infer which of their points are likely to lie in their overlap region.

There are a number of directions in which \acro\ could be extended. At present it is tightly coupled to fully convolutional point cloud encoders, and relies on having a reasonable number of \emph{superpoints} in the bottleneck. This could be a limitation in scenarios where the point density is very uneven. It would also be interesting to explore how our overlap-attention module can be integrated into direct point cloud registration methods and other neural architectures that have to handle two inputs with low overlap, e.g. in image matching~\cite{sarlin2020superglue}. Finally, registration in the low-overlap regime is challenging and \acro\ cannot solve all the cases. A user study could provide a better understanding of how \acro\ compares to human operators.