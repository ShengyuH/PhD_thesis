\subsection{SDF-based volume rendering for active sensor}\label{sec:sup_sdf_vol_render}
In this section, we start by introducing the preliminary of NeRF~\cite{mildenhall2020nerf} following terminology as described in~\cite{tagliasacchi2022volume}. Then we provide the full derivation of the SDF-based volume rendering for active sensor. 

\paragraph{Density}
For a ray emitted from the origin $\origin \in \real^3$ towards direction $\dir \in \real^3$, the \textit{density} $\density_\zeta$ at range $\zeta$ indicates the likelihood of light interacting with particles at that point $\ray_\zeta = \origin + \zeta \dir$. This interaction can include absorption or scattering of light. In passive sensing, density $\density$ is a critical factor in determining how much light from the scene's illumination is likely to reach the sensor after passing through the medium.'

\paragraph{Transmittance} 
quantifies the likelihood of light traveling through a given portion of the medium without being scattered or absorbed. Density is closely tied to the transmittance function $\trans(\zeta)$, which indicates the probability of a ray traveling over the interval $[0, \zeta)$ without hitting any particles. Then the probability $\trans(\zeta {+} d\zeta)$ of \emph{not} hitting a particle when taking a differential step $d\zeta$ is equal to $\trans(\zeta)$, the likelihood of the ray reaching $\zeta$, times $(1 - d\zeta \cdot \density(\zeta))$, the probability of not hitting anything during the step:
% 
\begin{align}
\trans(\zeta+d\zeta) =& \trans(\zeta) \cdot (1 - d\zeta \cdot \density(\zeta))
\\
\frac{\trans(\zeta+d\zeta) - \trans(\zeta)}{d\zeta} \equiv& \trans'(\zeta) = -\trans(\zeta) \cdot \sigma(\zeta) 
\label{eq:derivative}
\end{align}
% 
We solve the differential equation as follows:
%
\begin{align}
\trans'(\zeta) &= -\trans(\zeta) \cdot \density(\zeta) \\
\frac{\trans'(\zeta)}{\trans(\zeta)} &= -\density(\zeta) \\
\int_a^b \frac{\trans'(\zeta)}{\trans(\zeta)} \; d\zeta &= -\int_a^b \density(\zeta) \; d\zeta \\
\left. \log \trans(\zeta) \right|_a^b &= -\int_a^b \density(\zeta) \; d\zeta \\
\trans(a \rightarrow b) \equiv \frac{\trans(b)}{\trans(a)} &= \exponential{-\int_a^b \density(\zeta) \; d\zeta}   
\end{align}
% 
Hence, for a ray segment between $\zeta_0$ and $\zeta$, transmittance is given by:

\begin{equation}
\trans_{\zeta_0 \rightarrow \zeta} \equiv \frac{\trans_{\zeta}}{\trans_{\zeta_0}} = exp({-\int_{\zeta_0}^\zeta \density_t dt})\;,
\label{eq:trans_ab}
\end{equation}
which leads to following factorization of the transmittance:
\begin{equation}
\trans_{\zeta} = \trans_{0 \rightarrow \zeta_0} \cdot \trans_{\zeta_0 \rightarrow \zeta}\;.
\label{eq:factor}
\end{equation}

\paragraph{Opacity}
Opacity is the complement of transmittance and represents the fraction of light that is either absorbed or scattered in the medium. In a homogeneous medium with constant density $\density$  the opacity for a segment $[\zeta_j, \zeta_{j+1}]$ of length $\Delta \zeta$ is given by $\opacity_{\zeta_j} = 1 - exp(-\density \cdot \Delta \zeta)$

\paragraph{SDF-based volume rendering for active sensor}
NeuS\cite{wang2021neus} derives the opaque density based on the SDF which is:
\begin{equation}
\begin{split}
\density_{\zeta_i} =&  \max\left(\frac{-\frac{d\Phi_s}{d\zeta_i}(f(\zeta_i))}{\Phi_s(f(\zeta_i))},0\right)\\
                  =& \max\left(\frac{-(\nabla f(\zeta_i)\cdot \mathbf{v})\phi_s(f(\zeta_i))}{\Phi_s(f(\zeta_i))}, 0\right)
\end{split}
\label{eq:sigmoid_density_}
\end{equation}
where $\Phi_s$ represents the Sigmoid function, $f$ is the SDF function that maps a range $\zeta$ to the SDF value of the point position $\origin + \dir * \zeta$. Note that the integral term is computed by
\begin{equation}
\int \frac{-(\nabla f(\zeta)\cdot \mathbf{v})\phi_s(f(\zeta))}{\Phi_s(f(\zeta))}d\zeta = -\ln(\Phi_s(f(\zeta))) + C,
\label{eq:intergration_density}
\end{equation}
We extend the density-based volume rendering for active sensor to SDF-based. Starting from the passive SDF-based volume rendering \cite{wang2021neus}, We substitute the density $\tilde{\density}$ with opaque density in \ref{eq:sigmoid_density_}
and evaluate the radiant power integrated from ray segment [a,b] with constant reflectivity $\reflectivity_a$.

Consider the case where $-(\nabla f(\zeta)\cdot \mathbf{v})>0$ within the ray segment $[a,b]$, we have
\begin{align}
P(a \rightarrow b)
&= \int_a^b \trans^2(a\rightarrow t) \cdot \tilde{\density}_t \cdot \reflectivity(t)  \; dt
\\
&= \reflectivity_a \int_a^b \trans^2(a\rightarrow t) \cdot \tilde{\density}_t \; dt
\\
&= \reflectivity_a \int_a^b \exponential{-\int_a^t 2\tilde{\density}(u) \; du} \cdot \tilde{\density}_t \; dt
\\
&= \reflectivity_a \int_a^b \exponential{-2\int_a^t \tilde{\density}(u) \; du} \cdot \tilde{\density}_t \; dt
\\
&= \reflectivity_a \int_a^b \exponential{\left. 2\ln(\Phi_s(f(u)))\right|_a^t} \cdot \tilde{\density}_t \; dt
\\
&= \reflectivity_a \int_a^b \exponential{2\ln(\Omega_t) - 2\ln(\Omega_a)} \cdot \tilde{\density}_t \; dt 
\\
&= \reflectivity_a \int_a^b \frac{{\Omega_t}^2}{{\Omega_a}^2} \cdot \tilde{\density}_t \; dt ~~~~\text{\textbf{let} $\Omega_x = \Phi_s(f(x))$}
\\
&= \frac{\reflectivity_a}{{\Omega_a}^2} \int_a^b {\Omega_t}^2 \cdot \tilde{\density}_t \; dt 
\\
&= \frac{\reflectivity_a}{{\Omega_a}^2} \int_a^b -\frac{d\Phi_s}{dt}(f(t)) \cdot \Phi_s(f(t)) \; dt 
\\
&= \frac{\reflectivity_a}{{\Omega_a}^2} ( \left. -\frac{1}{2}{\Phi_s(f(t))}^2 \right|_a^b) \\
&= \frac{\reflectivity_a}{{\Omega_a}^2} (\frac{1}{2}{\Phi_s(f(a))}^2 -\frac{1}{2}{\Phi_s(f(b))}^2 )\\
&= \frac{{\Phi_s(f(a))}^2 -{\Phi_s(f(b))}^2}{{2\Phi_s(f(a))}^2} \cdot \reflectivity_a 
\label{eq:homogeneous}
\end{align}

Consider the case where $-(\nabla f(\zeta)\cdot \mathbf{v})<0$ within the ray segment $[a,b]$, we have
\begin{align}
P(a \rightarrow b)
&= \int_a^b \trans^2(a\rightarrow t) \cdot \tilde{\density}_t \cdot \reflectivity(t)  \; dt
\\
&= \int_a^b \trans^2(a\rightarrow t) \cdot 0 \cdot \reflectivity(t)  \; dt
\\
&= 0
\end{align}
Hence we conclude 
\begin{align}
P(a \rightarrow b)
&= \max\left(\frac{{\Phi_s(f(a))}^2 -{\Phi_s(f(b))}^2}{{2\Phi_s(f(a))}^2},0\right) \cdot \reflectivity_a 
\end{align}

\paragraph{Volume rendering of piecewise constant data}
Combining the above, we can evaluate the volume rendering integral through a medium with piecewise constant reflectivity:
% 
\begin{align}
P(\zeta_{N+1}) &= \sum_{n=1}^N \int_{\zeta_n}^{\zeta_{n+1}} \trans^2(\zeta) \cdot \tilde{\density}_{\zeta} \cdot \reflectivity_{\zeta_n} \; d\zeta
\\
&= \sum_{n=1}^N \int_{\zeta_n}^{\zeta_{n+1}} \trans^2_{\zeta_n} \cdot \trans^2(\zeta_n \rightarrow \zeta) \cdot \tilde{\density}_{\zeta} \cdot \reflectivity_{\zeta_n} \; d\zeta 
\\
&= \sum_{n=1}^N \trans^2_{\zeta_n}  \int_{\zeta_n}^{\zeta_{n+1}} \trans^2(\zeta_n \rightarrow \zeta) \cdot \tilde{\density}_{\zeta} \cdot \reflectivity_{\zeta_n} \; d\zeta 
\\
&=\sum_{n=1}^N \trans^2_{\zeta_n} P(\zeta_n \rightarrow \zeta_{n+1})
\\
&= \sum_{n=1}^N \trans^2_{\zeta_n} \cdot \tilde{\weight}_{\zeta_n} \cdot \reflectivity_{\zeta_n}
\end{align}

where 
\begin{align}
\tilde{\weight}_{\zeta_n} \equiv \max\left(\frac{{\Phi_s(f(\zeta_n)}^2 -{\Phi_s(f(\zeta_{n+1}))}^2}{{2\Phi_s(f(\zeta_n))}^2},0\right)
\end{align}

The discrete accumulated transmittance $\trans$ can be calculated as follows:

Consider the case where $-(\nabla f(\zeta)\cdot \mathbf{v}) > 0$ in $[\zeta_n, \zeta_{n+1}]$: 
%
\begin{align}
\trans_{\zeta_n} 
&=\prod_{i=1}^{n-1}(\exp(-\int_{\zeta_n}^{\zeta_{n+1}}\tilde{\density}_\zeta \; d\zeta) \\
&= \prod_{i=1}^{n-1}(\frac{\Phi_s(f(\zeta_{n+1}))}{\Phi_s(f(\zeta_n))})\\
\trans^2_{\zeta_n}
&= \prod_{i=1}^{n-1}(\frac{{\Phi_s(f(\zeta_{n+1}))}^2}{{\Phi_s(f(\zeta_n))}^2})\\
&= \prod_{i=1}^{n-1}(1-2\tilde{\weight}_{\zeta_n})
\label{eq:dicrete_transmittance}
\end{align}

Consider the case where $-(\nabla f(\zeta)\cdot \mathbf{v}) < 0$ in $[\zeta_n, \zeta_{n+1}]$: 
\begin{align}
\trans_{\zeta_n} 
&=\prod_{i=1}^{n-1}(\exp(-\int_{\zeta_n}^{\zeta_{n+1}}\tilde{\density}_\zeta \; d\zeta) = \prod_{i=1}^{n-1}(1)
\\
\trans^2_{\zeta_n} &= \prod_{i=1}^{n-1}(1^2) = \prod_{i=1}^{n-1}(1-2\tilde{\weight}_{\zeta_n})
\end{align}
In conclusion, the radiant power can be reformulated as:

\begin{align}
P(\zeta_{N+1}) = \sum_{n=1}^N \trans^2_{\zeta_n} \cdot \tilde{\weight}_{\zeta_n} \cdot \reflectivity_{\zeta_n}
\label{eq:final_radiant2}
\end{align}
where $\trans^2_{\zeta_n} = \prod_{i=1}^{n-1}(1-2\tilde{\weight}_{\zeta_i})$


\paragraph{Depth volume rendering of piecewise constant data}
Note that $\tilde{\weight}_{\zeta_n} \in [0, 0.5], \trans^2_{\zeta_n} \in [0,1], \sum_{n=1}^N \trans^2_{\zeta_n} \cdot \tilde{\weight}_{\zeta_n} = 0.5$, for depth volumetric rendering, we have 
\begin{align}
    \zeta = \sum_{n=1}^N 2 \cdot \trans^2_{\zeta_n} \cdot \tilde{\weight}_{\zeta_n} \cdot \zeta_n
    =\sum_{n=1}^N w_n \cdot \zeta_n
    \label{eq:depth_render}
\end{align}
where $w_n = 2\tilde{\weight}_{\zeta_n} \cdot \prod_{i=1}^{n-1}(1-2\tilde{\weight}_{\zeta_i})$